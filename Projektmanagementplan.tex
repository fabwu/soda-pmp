\documentclass[11pt,a4paper]{scrreprt}

% Codierung
\usepackage{selinput}
\SelectInputMappings{
  adieresis={ä},
  germandbls={ß},
}
\usepackage[ngerman]{babel}
\usepackage[T1]{fontenc}
\usepackage{lmodern}

% Kopf-/Fusszeilen
\usepackage{scrlayer-scrpage}
\usepackage{lastpage}
\setkomafont{pageheadfoot}{\small\sffamily}
\pagestyle{scrheadings}
\lohead*{Message-Logger}
\cohead*{VSK}
\rohead*{Team 3}
\lofoot*{Projektmanagementplan}
\cofoot*{}
\rofoot*{Seite \thepage \hspace{1pt} von \pageref{LastPage}}

\usepackage{lipsum}

% Bilder
\usepackage{graphicx}

% Hyperlinks
\usepackage[hidelinks]{hyperref}

% Quellcode
\usepackage{listings}
\lstset{language=Java}
\lstloadlanguages{Java}								% language options
\lstset{
	basicstyle=\ttfamily\footnotesize,      		% fontstyle
	numbers=left,									% linenumber placement
	numberstyle=\tiny,								% linenumber fontsize
	numbersep=5pt,									% how far the line-numbers are from the code
	breaklines=true,								% print linebreaks
	backgroundcolor=\color{gray!10},				% background color
	commentstyle=\color{commentcolor},				% comment color
	morecomment=[s][\color{javadoccolor}]{/**}{*/}, % javadoc color
	keywordstyle=\color{keywordcolor},				% keyword color
	stringstyle=\color{stringcolor},				% string color
	showstringspaces=false,							% show spaces in a string
	frame=single,									% frame
	tabsize=4,										% tabstop
	rulecolor=\color{black},						% if not set, the frame-color may be changed on line-breaks within not-black text (e.g. comments (green here)) 
	title=\lstname,									% title is name of source file
	escapeinside={\%*}{*)},			 				% if you want to add LaTeX within your code
	inputencoding=utf8,								% encoding = utf8
	extendedchars=true,								% utf8 support 
	literate=										% utf8 support
	{á}{{\'a}}1 {é}{{\'e}}1 {í}{{\'i}}1 {ó}{{\'o}}1 {ú}{{\'u}}1
	{Á}{{\'A}}1 {É}{{\'E}}1 {Í}{{\'I}}1 {Ó}{{\'O}}1 {Ú}{{\'U}}1
	{à}{{\`a}}1 {è}{{\`e}}1 {ì}{{\`i}}1 {ò}{{\`o}}1 {ù}{{\`u}}1
	{À}{{\`A}}1 {È}{{\'E}}1 {Ì}{{\`I}}1 {Ò}{{\`O}}1 {Ù}{{\`U}}1
	{ä}{{\"a}}1 {ë}{{\"e}}1 {ï}{{\"i}}1 {ö}{{\"o}}1 {ü}{{\"u}}1
	{Ä}{{\"A}}1 {Ë}{{\"E}}1 {Ï}{{\"I}}1 {Ö}{{\"O}}1 {Ü}{{\"U}}1
	{â}{{\^a}}1 {ê}{{\^e}}1 {î}{{\^i}}1 {ô}{{\^o}}1 {û}{{\^u}}1
	{Â}{{\^A}}1 {Ê}{{\^E}}1 {Î}{{\^I}}1 {Ô}{{\^O}}1 {Û}{{\^U}}1
	{œ}{{\oe}}1 {Œ}{{\OE}}1 {æ}{{\ae}}1 {Æ}{{\AE}}1 {ß}{{\ss}}1
	{ç}{{\c c}}1 {Ç}{{\c C}}1 {ø}{{\o}}1 {å}{{\r a}}1 {Å}{{\r A}}1
	{€}{{\EUR}}1 {£}{{\pounds}}1
}

\begin{document}
\titlehead{Hochschule Luzern \\ 
	Technik \& Architektur}
\subject{Projektmanagementplan}
\title{Message-Logger}
\subtitle{Team 3}
\author{Fabian Wüthrich \\ 
	Simon Erni \\ 
	Stefan Winterberger \\ 
	Alain Studhalter}
\date{\today}
\publishers{Verteilte Systeme \& Komponenten}

\maketitle

\tableofcontents

\chapter{Projektorganisation}

\section{Versionsverwaltung}

\begin{table}[h!]
	\renewcommand{\arraystretch}{1.5}
	\begin{tabular}{|c|c|c|p{7cm}|c|}
		\hline
		Version & Datum & Autor & Bemerkungen & Status \\ \hline
		0.1 & 16.02.2015 & F. Wüthrich & Ersterstellung & freigegeben \\ \hline
	\end{tabular}
\end{table}

\section{Organisationsplan, Rollen \& Zuständigkeiten}

\section{Projektstrukturplan}

\chapter{Projektführung}

\section{Rahmenplan}

\section{Projektkontrolle}

\section{Risikomanagement}

\section{Projektabschluss}

\chapter{Projektunterstützung}

\section{Tools für Entwicklung, Test und Abnahme}

\section{Konfigurationsmanagement}

\section{Reviewplanung}

\section{Abnahmeplanung}

\chapter{Testplan}

\section{Testdesign \& Abläufe}

\section{Testfälle}

\appendix

\chapter{Anhang}

\section{Sprintpläne}

\section{Meilensteinberichte}

\section{Reviewprotokolle}

\end{document}